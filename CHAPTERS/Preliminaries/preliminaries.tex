\chapter{Preliminaries and Notation} \label{chap:prelim}
\section{Set Theory} \label{sec:set}
//TODO: add later $\mathds{N}$, linear order $>_L$
\section{Graph Theory}  \label{sec:graph}
//TODO: add later. neighbours, $N_G$$(v)$
\subsection{Dominating Set}
//TODO:add later ds , dominated, non-dominated
\section{Complexity Theory} \label{sec:complex}
\subsection{Decision Problems} \label{subsec:decision}
\subsection{NP,NP-Hard, NP-complete}
//TODO:add later
\subsection{Growth Rate of Function} \label{subsec:growth}
\subsection{Fixed Parameter Tractability} \label{subsec:fpt}
//TODO: add later
\subsection{W-Hierarchy} \label{subsec:whierarchy}
//TODO:add later
\subsection{Kernelization} \label{subsec:kern}
\section{Reduction Rules} \label{sec:reduce}
\subsection{Crown Reduction Rule} \label{subsubsec:crown}
\section{\large{D}\normalsize{OMINATING} \Large{S}\normalsize{ET}} \label{subsec:comds}
Dominating set is one of natural properties of graphs, while \large{D}\normalsize{OMINATING} \Large{S}\normalsize{ET} problem is one of complex problems studied in complexity theory. \large{D}\normalsize{OMINATING} \Large{S}\normalsize{ET} problem is categorized in $NP-complete$ class \cite{garey1979}.
\begin{dproblem}
{\sc \large{D}\normalsize{OMINATING} \Large{S}\normalsize{ET}}\\
\instance A graph $G=(V,E)$ and $k \in \mathds{N}$.\\
\ques Is there a dominating set $D \subseteq V$ for $G$ such that $|D| \leqslant k$?\\
\cite{garey1979}
\end{dproblem}

\vskip -20pt
\large{D}\normalsize{OMINATING} \Large{S}\normalsize{ET} has been proved to be a $W[2]-complete$ problem by Downey and Fellows in 1995 \cite{downey1995}. In another words, this problem is not a $FPT$ problem and does not have kernel. Nevertheless, the incremental edition of this problem, $\large{I}\normalsize{NCREEMENTAL}\quad \large{D}\normalsize{OMINATING}\quad \Large{S}\normalsize{ET}$ problem, can be classified as a $FPT$ problem \cite{downey2014}.
\\
\\



