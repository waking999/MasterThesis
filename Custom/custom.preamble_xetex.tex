
%------------------------Latex Packages-----------------------------------
%\documentclass[a4paper,12pt,twoside,xcolor=pdftex,table]{book}
\documentclass[a4paper,12pt,twoside,xcolor=pstex,table,plainpages=false]{book}%,plainpages=false
%\documentclass[a4paper,12pt,twoside,table]{book}
%\documentclass[a4paper,12pt,twoside]{ulect-l}
\usepackage{threeparttable}
\usepackage{amsopn,amssymb,ifthen,amsmath}
\usepackage[lmargin=3cm,rmargin=2cm]{geometry}
%\usepackage{subfigure,epsfig}
\usepackage{subfigure,epsfig,calc}
\usepackage{graphics}
\graphicspath{{Graphs/}}
%\usepackage[dvips]{graphics}
%\usepackage[pdftex]{graphicx}
\usepackage{makeidx}
%\usepackage{fancyheadings}
\usepackage{fancyhdr}
\usepackage{fancyhdr}
\usepackage{mparhack}
\usepackage{theorem}
\usepackage{tabularx}
\usepackage{mathrsfs}
\usepackage{psfrag}
\usepackage{oldgerm}
\usepackage[boxed]{algorithm}
\usepackage[noend]{algorithmic}
\usepackage{color}
%\usepackage{verbatim}
\usepackage{setspace}
\usepackage{pstricks,pst-node,pst-text}
%\usepackage[plainpages=false]{hyperref}

\usepackage{multirow}
\usepackage{amsfonts}
\usepackage{dsfont}


%------------------------------------------------------------------------------

%=======================================
% More Latex Packages I need
%=======================================

\usepackage{xspace}
\usepackage{rotating}

%==============================================
% The following lines are suggested on the IFAC
% web page in order to use only standard PostScript fonts
% with type 1 encoding.
%==============================================

\usepackage{times}
\usepackage[T1]{fontenc}
\usepackage{textcomp}
\usepackage{Sweave}
\usepackage{mathpazo}
\usepackage{fontspec}
\usepackage{xunicode}
\usepackage{xltxtra}
%\usepackage{marginnote}
%\usepackage{pdfmarginpar}
\usepackage{mparhack}
%should add hyperlinks to bibliography
%\usepackage{newalg}
\usepackage{listings}
\usepackage{xcolor}
\usepackage{listings}
\usepackage{color}

\definecolor{mygreen}{rgb}{0,0.6,0}
\definecolor{mygray}{rgb}{0.5,0.5,0.5}
\definecolor{mymauve}{rgb}{0.58,0,0.82}

\definecolor{DarkPurple}{rgb}{0.4,0.1,0.4}
\definecolor{DarkCyan}{rgb}{0.0,0.5,0.4}
\definecolor{LightLime}{rgb}{0.4,0.6,0.5}
\definecolor{Blue}{rgb}{0.0,0.0,1.0}

\lstset{ %
language=Java,
  backgroundcolor=\color{white},   % choose the background color; you must add \usepackage{color} or \usepackage{xcolor}
  basicstyle=\footnotesize,        % the size of the fonts that are used for the code
  breakatwhitespace=false,         % sets if automatic breaks should only happen at whitespace
  breaklines=true,                 % sets automatic line breaking
  captionpos=b,                    % sets the caption-position to bottom
  commentstyle=\color{mygreen},    % comment style
  deletekeywords={...},            % if you want to delete keywords from the given language
  escapeinside={\%*}{*)},          % if you want to add LaTeX within your code
  extendedchars=true,              % lets you use non-ASCII characters; for 8-bits encodings only, does not work with UTF-8
  frame=single,                    % adds a frame around the code
  keepspaces=true,                 % keeps spaces in text, useful for keeping indentation of code (possibly needs columns=flexible)
  keywordstyle=\color{blue},       % keyword style
  language=Octave,                 % the language of the code
  morekeywords={*,...},            % if you want to add more keywords to the set
  numbers=left,                    % where to put the line-numbers; possible values are (none, left, right)
  numbersep=5pt,                   % how far the line-numbers are from the code
  numberstyle=\tiny\color{mygray}, % the style that is used for the line-numbers
  rulecolor=\color{black},         % if not set, the frame-color may be changed on line-breaks within not-black text (e.g. comments (green here))
  showspaces=false,                % show spaces everywhere adding particular underscores; it overrides 'showstringspaces'
  showstringspaces=false,          % underline spaces within strings only
  showtabs=false,                  % show tabs within strings adding particular underscores
  stepnumber=2,                    % the step between two line-numbers. If it's 1, each line will be numbered
  stringstyle=\color{mymauve},     % string literal style
  tabsize=2,                       % sets default tabsize to 2 spaces
  title=\lstname                   % show the filename of files included with \lstinputlisting; also try caption instead of title
}

\lstdefinestyle{customJava}
{
language=Java,
numbers=left,
firstnumber=1,
stepnumber=5,
numberfirstline,
numberstyle=\tiny\sffamily,
tabsize=5,
captionpos=b,
aboveskip=1em,
belowskip=1em,
columns=flexible,
xleftmargin=2em,
xrightmargin=1em,
frame=single,
frameround=tttt,
commentstyle=\itshape\color{LightLime},
keywordstyle=\bfseries\color{DarkPurple},
basicstyle=\footnotesize\ttfamily,
stringstyle=\color{Blue},
showstringspaces=false,
}



\newcommand{\includecode}[2][Java]{\lstinputlisting[caption=#2, escapechar=, style=custom#1]{#2}}

%GET THIS WORKING TO GET HYPERLINKS TO WORK --- SLOWS DOWN COMPILE A LOT
%\definecolor{webgreen}{rgb}{0, 0.5, 0} % less intense green
%\definecolor{webblue}{rgb}{0, 0, 0.5} % less intense blue
\definecolor{webred}{rgb}{0.5, 0, 0} % less intense red
\definecolor{MyDarkBlue}{rgb}{0.1,0,0.35}
\usepackage[%ps2pdf,
		backref=page,
		pdfpagelabels,
		pagebackref=true,
		hyperindex=true,
                bookmarks=true,
                bookmarksnumbered=false, % true means bookmarks in left window are numbered
                bookmarksopen=false, % true means only level 1 are displayed.
                colorlinks=true,
                plainpages=false,
                citecolor=MyDarkBlue,
                linkcolor=MyDarkBlue,
                xetex]{hyperref}
%\usepackage[hyperpageref,plainpages=false]{backref} %Page 15 backref manual
\defaultfontfeatures{Mapping=tex-text}
%%%%%%%%%%%%
%Required by Newcastle UNI
%\onehalfspacing
\doublespacing
\setmainfont{Palatino Linotype}
%\newfontfamily\sectionfont{Hoefler Text}
\newfontfamily\sectionfont{Palatino Linotype}

%%%%%%%%%%%%


%
%%%%%%%%%%%%%%%%%%%%%%%%%%%%%%%%%%%%%%%%%%%%%%%%%%%%
%
\newcommand{\comment}[2][Ref]{{\marginpar{Comment {#1}-- {#2}}}}
%\newcommand{\comment}[2][Ref]{{\pdfmarginpar{Comment {#1}-- {#2}}}}

%\newcommand{\comment}[2][Ref]{{\marginnote{Comment {#1}-- {#2}}}}

%==============================================
% Some handy definitions
%==============================================

\newcommand{\eg}        {\emph{e.g.},\xspace}
\newcommand{\ie}        {\emph{i.e.},\xspace}

\newcommand{\siso}      {\textsc{siso}\xspace}
\newcommand{\mimo}      {\textsc{mimo}\xspace}
\newcommand{\sfaw}   {\textsc{sfaw}\xspace}
\newcommand{\qp}   {\textsc{qp}\xspace}
\newcommand{\mpc}   {\textsc{mpc}\xspace}
\newcommand{\dft}   {\textsc{dft}\xspace}
\newcommand{\svd}   {\textsc{svd}\xspace}
\newcommand{\cd}   {\textsc{cd}\xspace}
\newcommand{\rhc}   {\textsc{rhc}\xspace}
\newcommand{\lqr}   {\textsc{lqr}\xspace}

\newcommand{\alert}   {\red}
\newcommand{\smallO}[1]{o{\left(#1\right)}}
\newcommand{\smallOn}{o}

\DeclareMathOperator{\trace}{trace}
\DeclareMathOperator{\diag}{diag}

%==============================================
% My definitions for Theorems and the like
%==============================================

\newtheorem{thm}{Theorem}[section]
\newtheorem{cor}[thm]{Corollary}
%\newtheorem{lem}[thm]{Lemma}
\newtheorem{fact}[thm]{Fact}
\newtheorem{prop}[thm]{Proposition}
\newtheorem{assu}{Assumption}
\newtheorem{examp}[subsection]{Example}
\newtheorem{remk}[subsection]{Remark}
\newtheorem{defn}[thm]{Definition}

\newtheorem{parra}[subsection]{Paraphrase \textcolor{webblue}}
\def\qed{\hfill{$\Box$}}
\def\yes{\textsc{yes}\;}
\def\no{\textsc{no}\;}
\def\yesh{\textsc{yes}}
\def\noh{\textsc{no}}
\def\kcluster{{\sc $k$-Cluster Edit}\;}
%===========================================
% Alternative Theorem styles required for other standards
%===========================================
\newtheorem{lemma}[thm]{Lemma}
%================================================
% Abbreviation and layout definitions
%================================================
\newcommand{\cedit}{{\scshape Cluster Editing}\;}
\newcommand{\ceditc}{{\scshape Cluster Editing}}
%\newcommand{\cedit}{{\sc Cluster Editing}\;}
%\newcommand{\ceditc}{{\sc Cluster Editing}}
\newcommand{\vc}{{\scshape Vertex Cover}\;}
\newcommand{\vcc}{{\scshape Vertex Cover}}
\newcommand{\fpt}{{$\mathcal{FPT}$}\;}
\newcommand{\fptc}{{$\mathcal{FPT}$}}
%\newcommand{\N}{{\mathbb{N}}\;}
%\newcommand{\Poly}{{$\mathbb{P}$}\;}
%\newcommand{\NP}{{$\mathbb{NP}$}\;}
%\newcommand{\APX}{{$\mathbb{APX}$}\;}
\newcommand{\N}{{\mathcal{N}}\;}
\newcommand{\Poly}{{$\mathcal{P}oly$}\;}
\newcommand{\NP}{{$\mathcal{N\!P}$}\;}
\newcommand{\NPO}{{$\mathcal{N\!PO}$}\;}
\newcommand{\APX}{{$\mathcal{AP\!X}$}\;}
\newcommand{\PTAS}{{$\mathcal{PTAS}$}\;}
\newcommand{\PTASc}{{$\mathcal{PTAS}$}}
\newcommand{\EPTAS}{{$\mathcal{EPTAS}$}}
\newcommand{\FPTAS}{{$\mathcal{FPTAS}$}}
\newcommand{\NPc}{{$\mathcal{N\!P}$}}
\newcommand{\APXc}{{$\mathcal{A\kern-0.1em P\kern-0.2em X}$}}
\newcommand{\vst}{\vspace{3mm}}
\newcommand{\hsd}{\hspace{10mm}}
\newcommand{\field}[1]{\mathbb{#1}}
\newcommand{\aidx}[1]{{\em {#1}} \index{#1}}
%\newcommand{\N}{\field{N}}
%\newcommand{\fpt}{{\sc Fpt }\;}
\newcommand{\Oh}{{\mathcal O}}
\newcommand{\etal}{{\em{et al.}\;}}
\newcommand{\etals}{{\em{et al.'s}} }
\newcommand{\ind}[1]{\langle #1 \rangle}

%=================================
%PSTRICK MACROS
\newcommand{\SB}[1]{\psshadowbox[linecolor=yellow]{#1}}

%==================================
% Bibliography definition
%==================================

%\bibliographystyle{ifac}
%\usepackage[dcucite]{harvard} BRAKES CITATIONS IF USED

%==============================================
% Defining the path for figures
% (This is handy when figures are in different
% directories)
%==============================================

%\graphicspath{{Chapters/Application/Fig/}, {Chapters/Asymptotic_svs/Fig/}}


%------------------------Page set-up-----------------------------------------
%\renewcommand{\baselinestretch}{1.5}
\setlength{\hoffset}{-1in}
\setlength{\oddsidemargin}{3cm}
\setlength{\evensidemargin}{3cm}
\setlength{\voffset}{3cm - 1in}
\setlength{\topmargin}{0cm}
\setlength{\headheight}{15pt}
%\setlength{\headsep}{0cm}
\setlength{\marginparwidth}{2.5cm}
\setlength{\marginparsep}{0cm}
\setlength{\marginparpush}{0cm}
%\setlength{\footskip}{0cm}
\setlength{\textheight}{242mm - \headheight - \headsep - \footskip}
\setlength{\textwidth}{14cm}
\setlength{\parindent}{0cm}
\setlength{\parskip}{0.75\baselineskip}

%------------------------------------------------------------------------------

%------------------------Setup headings and spacing---------------------
\makeatletter
\renewcommand{\section}{\@startsection%
  {section}% name
  {1}% level
  {0mm}% indent
  {0.5\baselineskip}% beforeskip
  {0.1mm}% afterskip
  {\sectionfont\Large\bfseries}% style
}
\renewcommand{\subsection}{\@startsection%
  {subsection}% name
  {2}% level
  {0mm}% indent
  {0.5\baselineskip}% beforeskip
  {0.1mm}% afterskip
  {\sectionfont\large\bfseries}% style
}
\makeatother
%------------------------------------------------------------------------------

%------------------------Fancy headings-----------------------------------
\pagestyle{fancyplain}
\addtolength{\headwidth}{\marginparsep}
%\addtolength{\headwidth}{\marginparwidth}
\renewcommand{\chaptermark}[1]{\markboth{#1}{}}
\renewcommand{\sectionmark}[1]{\markright{\thesection\ #1}}
\lhead[\fancyplain{}{\bfseries\thepage}]{\fancyplain{}{\bfseries\rightmark}}
\rhead[\fancyplain{}{\bfseries\leftmark}]{\fancyplain{}{\bfseries\thepage}}
\cfoot{}
%------------------------------------------------------------------------------

% Modify the first page of the chapters: no chapter word, big chapter
% number, title adjusted to right, and a line below.
%-----------------------------------------------------------------------
\makeatletter
\def\@makechapterhead#1{%
  {\parindent \z@\raggedleft
    \ifnum \c@secnumdepth >\m@ne
    %{\fontsize{20pt}{30pt}\selectfont\Huge\textsc{\textbf{\thechapter}}}
    {\sectionfont\Huge{\thechapter}}
    \fi
    \rule{\columnwidth}{0pt} \par
    {\sectionfont\Huge\textsc{#1}\par}
    \nobreak
    \vskip 70\p@
    }}
\def\@makeschapterhead#1{%
  {\parindent\z@\raggedleft
    {\sectionfont\Huge\textsc{#1}\par}
    \nobreak
    \vskip 70\p@
    }}
\makeatother
%-----------------------------------------------------------------------

% Modify the caption definition to make text italic and smaller than normalsize text
%-----------------------------------------------------------------------
\makeatletter
\newsavebox{\capbox}
\renewcommand{\@makecaption}[2]{%
  \vspace{0pt}\sbox{\capbox}{\textsf{\textbf{#1}}: \textsf{\small{#2}}}%
  \ifthenelse{\lengthtest{\wd\capbox > 0.8\linewidth}}%
    {\center{\parbox[t]{0.8\linewidth}{\textsf{\textbf{#1}}: \textsf{\small{#2}}}}}%
    {\begin{center}\textsf{\textbf{#1}}: \textsf{\small{#2}}\end{center}}%
  } \makeatother
%-----------------------------------------------------------------------


% Pages and Fancyheadings stuff
%-----------------------------------------------------------------------
\pagestyle{fancy}
% \pagestyle{fancyplain}
%\setlength{\headrulewidth}{0.1pt} \setlength{\footrulewidth}{0pt}
\newcommand{\clearemptydoublepage}{\newpage\thispagestyle{empty}\cleardoublepage}
\cfoot{}
% Chapter
\renewcommand{\chaptermark}[1]{\markboth{\sectionfont\thechapter.\ #1}{}}
% % Section
\renewcommand{\sectionmark}[1]{\markright{\sectionfont\thesection\ #1}}
\fancyhf{}
\fancyhead[LE,RO]{\bfseries\thepage}
\fancyhead[LO]{\nouppercase{\bfseries\rightmark}}
\fancyhead[RE]{\nouppercase{\scshape\leftmark}}
% \lhead[\fancyplain{}{\bf\thepage}]{\fancyplain{}{\bf\rightmark}}
% \rhead[\fancyplain{}{\bf\leftmark}]{\fancyplain{}{\bf\thepage}}
\fancypagestyle{plain}{%
\fancyhf{} % clear all header and footer fields
\renewcommand{\headrulewidth}{0pt}
\renewcommand{\footrulewidth}{0pt}}
%-----------------------------------------------------------------------

%---------------------------New Commands-------------------------------
\newcommand{\nc}{\newcommand}
\nc{\nenv}{\newenvironment}
\nc{\optprobmin}[8]{%
                                %\begin{equation*}%
  \ifthenelse{#1 > 0}{%
    \begin{alignedat}{3}%
      ({\cal #2}_{#3}):\qquad\qquad%
      \underset{#4}{\textnormal{min}}&&\quad&#5\\%
      \textnormal{s.t.}&&\quad%
      \ifthenelse{#1 = 3}{&#6\\&&&#7\\&&&#8}{}%
      \ifthenelse{#1 = 2}{&#6\\&&&#7}{}%
      \ifthenelse{#1 = 1}{&#6}{}%
    \end{alignedat}%
  }{({\cal #2}_{#3}):\qquad\qquad \underset{#4}{\textnormal{min}}\ #5}
                                %\end{equation*}%
}
\nc{\optprobargmin}[9]{%
                                %\begin{equation*}%
  \ifthenelse{#1 > 0}{%
    \begin{alignedat}{3}%
        ({\cal #2}_{#3}):\qquad\qquad #4 \wideq \textnormal{arg}\,%
        \underset{#5}{\textnormal{min}}&&\quad&#6\\%
      \textnormal{s.t.}&&\quad%
      \ifthenelse{#1 = 3}{&#7\\&&&#8\\&&&#9}{}%
      \ifthenelse{#1 = 2}{&#7\\&&&#8}{}%
      \ifthenelse{#1 = 1}{&#7}{}%
    \end{alignedat}%
  }{({\cal #2}_{#3}):\qquad\qquad #4 \wideq \textnormal{arg}\,%
    \underset{#5}{\textnormal{min}}\ #6}
                                %\end{equation*}%
}
\nc{\optprobmax}[8]{%
  %\begin{equation*}%
  \begin{alignedat}{3}%
    ({\cal #2}_{#3}):\qquad\qquad%
    \underset{#4}{\textnormal{max}}&&\quad&#5\\%
    \textnormal{s.t.}&&\quad%
    \ifthenelse{#1 = 3}{&#6\\&&&#7\\&&&#8}{}%
    \ifthenelse{#1 = 2}{&#6\\&&&#7}{}%
    \ifthenelse{#1 = 1}{&#6}{}%
  \end{alignedat}%
%\end{equation*}%
}
\nc{\mbf}{\mathbf}
\nc{\R}{\Bbb{R}}
%\nc{\U}{\Bbb{U}}
\nc{\X}{\Bbb{X}}
\nc{\intr}[1]{\textnormal{int}#1}
\nc{\mpart}[2]{\frac{\partial #1}{\partial #2}}
\nc{\sfrac}[2]{{\textstyle\frac{#1}{#2}}}
\nc{\mprb}[2][]{\ensuremath{({\cal #2}_{#1})}}
\nc{\nser}[3]{#1_{#2},\ldots,#1_{#3}}
\nc{\nserf}[4]{#1(#2_{#3}),\ldots,#1(#2_{#4})}
\nc{\inprod}[2]{\langle#1,#2\rangle}
\nc{\ndthm}{\hspace*{\fill} $\scriptstyle{\Box}$}
\nc{\vndthm}{\vspace{-\baselineskip} \ndthm}
\nc{\wideq}[1][=]{\ensuremath{\ \ #1\ \ }}
\nc{\xs}{{\mathscr X}}
\nc{\us}{{\mathscr U}}
\nc{\Seq}{\Bbb{S}}
\nc{\range}[1]{\textnormal{ran}(#1)}
\nc{\domain}[1]{\textnormal{dom}(#1)}
%------------------------------------------------------------------------------
%
%---------------------------New Theorems-------------------------------
% Adrian's definitions for Theorems and the like :

\nc{\nt}{\newtheorem}
%\nt{theorem}{Theorem}[section]
%\nt{property}{Property}[section]
%\nt{remark}{Remark}[section]
%\nt{lemma}{Lemma}[section]
\nt{corollary}{Corollary}[section]
\nt{definition}{Definition}[section]
%\nt{result}{Result}[section]
%\nt{example}{Example}[section]
\nt{observation}{Observation}[section]
\nt{rr}{Reduction Rule}[section]
\nt{claim}{Claim}[section]
%\nt{process}{Process}[section]
%\nt{proposition}{Proposition}[section]
%\nt{assumption}{Assumption}[section]
%\nt{problem}{Problem}[section]
%\nt{algorithm}{Algorithm}[section]
\newenvironment{claimproof}[1]
{\begin{quote}\emph{Proof of Claim #1.}} {\qed\end{quote}}
\newtheorem{redrule}{Reduction Rule}
\newenvironment{redruleproof}[1]
{\begin{quote}\emph{\textbf{Proof of Reduction Rule #1.}}}
{\qed\end{quote}}

%------------------------------------------------------------------------------
%
%---------------------------New Environments-------------------------------
\def\qed{\ifmmode\hspace*{\fill}\Box\else\unskip\nobreak\hspace*{\fill}$\Box$\fi}
%\hspace*{\fill} $\Box$
%\def\qed{\hspace*{\fill} $\Box$}
\nenv{proof}[0]{\textit{\textbf{Proof. }}}{\hspace*{\fill} $\Box$}
\nenv{proplist}[1]{\begin{list}{}{%
      \renewcommand{\makelabel}[1]{(#1.##1)}%
      \settowidth{\labelwidth}{(#1.2)}%
      \setlength{\itemindent}{0mm}%
      \setlength{\labelsep}{5mm}%
      \setlength{\leftmargin}{2cm}%
      \setlength{\rightmargin}{5mm}%
      \setlength{\topsep}{0.5\baselineskip}%
      \setlength{\parskip}{0mm}%
      \setlength{\partopsep}{0mm}%
    }}{\end{list}}

\hyphenpenalty 4000
\tolerance 1000
%\XeTeXlinebreaklocale "en_US"
%\XeTeXlinebreakpenalty 230
%\XeTeXlinebreakskip=0pt plus 15pt
